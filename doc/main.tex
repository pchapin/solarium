%%%%%%%%%%%%%%%%%%%%%%%%%%%%%%%%%%%%%%%%%%%%%%%%%%%%%%%%%%%%%%%%%%%%%%%%%%%%
% FILE   : main.tex
% AUTHOR : (C) Copyright 2024 by Peter Chapin
% SUBJECT: Document describing the solar system simulator.
%
% Send comments or bug reports to:
%
%       Peter Chapin
%       Computer Information Systems Department
%       Vermont State University
%       Williston, VT 05061
%       spicacality@kelseymountain.org
%%%%%%%%%%%%%%%%%%%%%%%%%%%%%%%%%%%%%%%%%%%%%%%%%%%%%%%%%%%%%%%%%%%%%%%%%%%%

%+++++++++++++++++++++++++++++++++
% Preamble and global declarations
%+++++++++++++++++++++++++++++++++
\documentclass{article}

\usepackage{graphicx}
\usepackage{listings}
\usepackage{url}
\usepackage{hyperref}

% The following are settings for the listings package.
\lstset{language=C,
        basicstyle=\small,
        stringstyle=\ttfamily,
        commentstyle=\ttfamily,
        xleftmargin=0.25in,
        showstringspaces=false}

\setlength{\parindent}{0em}
\setlength{\parskip}{1.75ex plus0.5ex minus0.5ex}

\newcommand{\filename}[1]{\texttt{#1}}

%++++++++++++++++++++
% The document itself
%++++++++++++++++++++
\begin{document}

%-----------------------
% Title page information
%-----------------------
\title{Solarium: A Solar System Simulator}
\author{Peter Chapin}
\date{January 7, 2024}
\maketitle

\section{Introduction}
\label{sec:introduction}

The dynamics of solar systems is an interesting research topic in astronomy. Although many solar
systems exhibit a high degree of stability even over billions of years, that is not universally
the case. When many bodies interact gravitationally, chaotic behavior can arise causing planets
to be ejected from the system or even to collide. There is evidence that this sort of chaos even
occurred in our own solar system early in its history\footnote{It is currently believed that a
Mars sized planet collided with the Earth causing the formation of the Moon.}

The purpose of this program is to simulate the motion of objects in a solar system. The
simulation is eventually intended to run for millions (or even billions) of simulated years to
explore long term solar system stability. We will assume that a solar system is composed of a
finite number of gravitationally interacting ``point'' masses. We will also consider each object
to be a sphere with a fixed size so that collisions can be detected and counted.

This model of the solar system is reasonably accurate for our purposes. The effects of
interplanetary gas and dust as well as exotic ``dark matter'' or ``dark energy'' should not be
significant on the scales of interest here. In this model the motion of the objects in the solar
system can be completely described by the response to their mutual gravitational attraction.

It should be said, that although it is hoped that the program described here will produce
reasonable and interesting results, the primary goal of this exercise is to demonstrate various
parallel programming techniques in the context of a typical high performance computing
application.

The rest of this document is organized as follows. Section~\ref{sec:computation} describes the
required computation in more detail along with some mathematical background.
Section~\ref{sec:preliminaries} discusses some preliminary matters such as what system of units
should be used and how the input and output of the computation should be prepared and handled. A
sequential program that solves the problem is presented and discussed in detail in
Section~\ref{sec:sequential}. Various issues surrounding how to parallelize the computation,
along with related programming exercises are then presented in Section~\ref{sec:parallelizing}.
Finally, the document concludes with some benchmarks, a summary of results, and so forth.


\section{Computation}
\label{sec:computation}

Each object in the solar system has a location given by three coordinates. These coordinates
constitute the object's position vector. In addition each object has a velocity vector that
defines the object's motion. The position and velocity of an object are functions of time.

In our simulation time moves forward in discrete jumps called ``time steps.'' The object's
position and velocity are updated with each time step. The input of the computation consists of
three tables; one that gives the initial position of each object, another that gives the initial
velocity of each object, and a third that gives the mass of each object. The output of the
computation is a collection of tables with one table for each object giving the position of that
object for every time step of the simulation. Alternatively one could regard the output as a
single, large table with a row for each time step and a column for each object. Note that the
velocity of the objects are not output but velocities must be computed so that position can be
updated properly.

Real solar systems contain a large number of objects and have lifespans measured in billions of
years. A fully accurate simulation would thus create an immense amount of data. To make the
amount of data produced managable our simulation will track a much smaller number of objects and
use as large a time step as possible. Keeping the simulation accurate under these conditions is
a significant problem in numerical analysis. Although these matters are quite interesting they
are outside the scope of this exercise. We will build our software and then tinker with the
parameters to see how things behave.

In principle the physics involved in this simulation is very simple. Each object in our universe
exerts a gravitational attraction on every other object. To compute the motion of a object one
would follow the steps below.

\begin{enumerate}

\item Add up the graviational force on the object due to every other object in the universe. The
  result is a vector in three dimensions.

\item Since $F = ma$, the acceleration of the object can be computed by doing $F/m$ where $m$ is
  the mass of the object. The result is also a vector in three dimensions.

\item Since $a = dv/dt$, multiplying the acceleration vector by the length of a time step
  results in a vector giving the change in velocity over that time step. Add that change in
  velocity to the object's initial velocity vector to obtain its velocity vector at the end of
  the time step.

\item Since $v = dx/dt$, multiplying the velocity vector by the length of a time step results in
  a vector giving the change in position over that time step. Add that change in position to the
  object's initial position vector to obtain it's position vector at the end of the time step.

\item Output the object's new position vector. Repeat for all objects in the universe. Advance the
  master clock by one time step and repeat the entire computation.

\end{enumerate}

At the end of the computation we would have a table with one row for each time step and one
column for each object. In each table cell would be the position of a particular object at a
particular time. Scanning down a column would thus reveal the motion of a single object over
time. Scanning across a row would show the position of every object in the universe for a
particular time.

You may recognize this process as that of solving a differential equation using numerical
integration. In fact, the method I describe above for evaluating the integrals is quite
simplistic. Eventually we should replace this method with something more sophisticated. However,
if our primary goal is to explore parallel programming it is not critical to build a well
structured numerical computation.


\section{Preliminaries}
\label{sec:preliminaries}

\subsection{Units and Scaling}

\textit{TODO: Finish me!}

\subsection{Input Data}

\textit{TODO: Finish me!}

\subsection{Visualization}

Once the results have been calculated, the next step is visualizing those results. A large table
of values is not easy for humans to interpret. Instead, a program should be used that renders
the information in some easy to understand manner (typically graphically). This could be done
off-line and might even be best done by prebuilt tools. For example, MATLAB might be able to
help with the visualization. However, if necessary we can write our own program(s).

I'm imagining a program that displays each object in the universe as a dot on the screen. It
could then animate the image so that we could watch the objects moving in accordance with the
results of the calculation. Other forms of visualization might also be possible (chart the
motion of the center of mass perhaps?).


\section{Sequential}
\label{sec:sequential}

The first step in this project is to create a sequential (single threaded) program that solves
this problem. This program can be used as a performance baseline against which the parallel
programs can be compared. Ideally a parallel program with $n$ processing elements (threads,
cluster nodes, etc.) would run $n$ times faster than the sequential program.

The sequential program will be written in the C programming language. Many high performance
computations are written in Fortran, a language optimized for this kind of application. Fortran
programs can often outperform equivalent C programs by 10--20\%.

The sequential program is called \filename{Serial}. The program contains three parts.
\begin{enumerate}
\item Read initial inputs.
\item Do the computation (loop over all time steps, etc.).
\item Write the outputs.
\end{enumerate}

Because \filename{Serial} can produce a considerable amount of data for each time step, a
version of \filename{Serial} with practical limits on memory consumption will need to output
results as it goes along. This means that timing data must be gathered inside the program so
that the computation time can be clearly distinguished from the disk I/O time\footnote{Both
times are interesting, of course. Our parallel machine can do parallel disk access and so it
should be able to accelerate both computation and I/O operations.}.

% This paragraph should probably go into a separate section.
One interesting question is: can the program go fast enough to allow real time visualization of
the results? That is, can it compute results as fast as they are needed by the visualization
software? Most major simulations are not able to do this when run on realistic problem sizes.
However, if the simulation can run that quickly it creates the possibility of making the
simulation interactive.

\subsection{Input}
\label{sec:sequential-input}

Each problem instance is described by a small text file that defines the size of the problem and
specifies the location of the other input files. \filename{Serial} takes the name of this
problem instance file as a command line argument; it is the only input file that needs to be
directly specified.

The format of the problem instance file is as a sequence of name=value pairs. Blank lines are
ignored. Leading and trailing white space is ignored. White space around the `=' character is
ignored. However, embedded white space in names or values is significant. Comments can be
introduced with a `\#' character; all text from that character to the end of the line is
ignored. What follows is a description of the various problem instance settings that can be
stored in the problem instance file. Except as otherwise noted, the settings can appear in any
order in the file. All settings must be present; there are no defaults.

\begin{description}
\item[Version] (integer) The version of the problem instance file in use. This item must be the
  first non-blank, non-comment line in the file. Currently, only version 1 is supported.
\item[Units] (string) An identifier that indicates what system of units are being used by the
  simulation. The allows values of this identifier are not specified here. However,
  \filename{serial} will abort with an error if it does not recognize the identifier used.
\item[N] (integer) The number of objects in the simulation. This value is used to specify the
  size of the input tables. $N > 0$ is required.
\item[MassTable] (string) The name of the file containing the object mass data.
\item[PositionTable] (string) The name of the file containing the initial positions of all the
  objects.
\item[VelocityTable] (string) The name of the file containing the initial velocity of all the
  objects.
\end{description}

The first object (object \#0) is treated in a special way by the visualization system. It is
intended to represent the sun/star that dominates the system. However, there is no special
handling of this object by the computations. Multiple star systems, or systems without stars,
can also be readily simulated.

Here is a sample problem instance file
\begin{verbatim}
# This is a sample file.

Version = 1   # Only version allowed. Must be first.
N=100
Units = mks

# Use a common mass table.
MassTable = C:\Documents and Settings\Peter\serial\mass1.txt

PositionTable = run1-position.txt
VelocityTable = run1-velocity.txt
\end{verbatim}

The input tables use a format similar to the problem instance file. They are text files with one
record on each line. Blank lines and comments introduced with `\#' are ignored. The first line
must be a Version setting with a value that agrees with the Version setting in the problem
instance file. Currently, only Version 1 is supported. The next two lines must contain a unit
setting and a value of $N$ that agree with what is in the problem instance file. This redundancy
ensures that one does not accidentally create a problem instance with, for example, an
inconsistent system of units.

The number of records in a table file must, naturally, agree with the specified value of $N$. If
it does not \filename{Serial} will display an error message and abort. If the number of records
is not correct \filename{Serial} assumes there is an error in the file and won't spend time
doing a run on bad data.

Each data line starts with an integer $k$ in the range $1 \le k \le N$. The lines must be
numbered in order; it is a fatal error if they are not. Following the integer is a comma
separated list of values. The mass table contains a single value on each line. The other two
tables contain $(x, y, z)$ coordinates of the initial position or velocity as appropriate. The
format of each value follows the usual conventions for floating point numbers. Extra spaces
separating the values are ignored.

Here is a sample velocity table for $N=4$
\begin{verbatim}
# This is a sample.

Version = 1
N = 4
Units = xyz

# The actual data, one (V_x, V_y, V_z) record for each star.
1  1.23e-3, -4.24e-2,  5.04e+1
2  8.79e-4,  5.63e-1,  8.88e+0
3  2.11e-2,  2.43e+2, -6.55e+1
4 -3.22e-1,  4.34e+1,  6.93e+1
\end{verbatim}

\textit{TODO: Finish me!}


\section{Parallelizing}
\label{sec:parallelizing}

\textit{TODO: Talk about breaking this problem down into a thread based concurrent solution.}

Obviously we would need to break this problem into pieces if we are to run it effectively on a
parallel machine. One obvious way of doing this could work pretty well.

Suppose the objects in the universe were partitioned into equal sized subsets with each node
given authority of one subset. For example, if we use $10$ nodes and simulate a universe of
$1000$ objects then each node would be authoritative over $1000/10 = 100$ objects.

The computation would proceed in two phases. The first phase would be the communication phase.
During that phase each node informs all other nodes the current positions of its objects. During
this phase no computations are done but the network connecting the nodes would be heavily used.
Following the communication phase would be the computational phase. During that phase each node
computes updated position information for the objects it has been assigned. When all nodes have
completed their updates, the master clock moves forward one time step and the process repeats
with a new communication phase.

It is interesting to consider the running time of this system. If the number of nodes is fixed
the time required for the communication phase is $O(n)$ where n is the number of objects in the
universe. For example, if the number of objects were to double, the size of each message in the
communication phase would also double and thus the communication time would double. However, the
computational phase runs in $O(n^2)$. If the number of objects doubles, each node will have
twice as many objects to consider and for each object, twice as many gravitational interactions
to compute.

Because the computational phase runs in $O(n^2)$ and the communication phase runs in $O(n)$
there will be some $n$, perhaps large, where more time is spent computing than communicating.
Thus the cluster's efficiency increase as $n$ increases. It is a happy circumstance that a large
$n$ also gives a more accurate simulation. Of course a large $n$ will also require a lot of time
to get results. It would be interesting to experiment with the value of $n$ to find out, for
example, at what point the cluster begins to run efficiently.

\textit{TODO: Also consider giving each node authority over a particular region of space.
  Consider the effect of ignoring ``distant'' objects.}


\textit{TODO: Add final sections!}

\end{document}
